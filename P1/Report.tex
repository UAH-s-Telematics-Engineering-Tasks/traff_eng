\documentclass[11pt]{article}

% Handle Spanish seamlessly!
\usepackage[utf8]{inputenc}
\usepackage[spanish]{babel}

% Needed for the multline environment
\usepackage{amsmath}

\usepackage{geometry}
\geometry{
    a4paper,
    left = 17mm,
    right = 17mm,
    top = 12mm,
    bottom = 20mm
}

\title{Acostumbrándonos al entorno \texttt{GNS3}}
\author{Pablo Collado Soto \\ \\ \textit{Ingeniería de Tráfico}}
\date{}

\begin{document}
    \maketitle

    \section{Introducción}
        En esta práctica vamos a intentar familiarizarnos con el entorno \texttt{GNS3} así como con \texttt{mgen}, herramienta con la que generamos tráfico para poner a prueba la red bajo estudio. Tenga en cuenta que los archivos que se han empleado para generar tráfico en el escenario, así como los archivos que recogen el tráfico real sobre la red se adjuntan en el repositorio asociado.\\

        En nuestro escenario contamos con un equipo \texttt{SRC} que generará tráfico que atravesará al router \texttt{R1} para llegar al equipo \texttt{DST}. Como cabría esperar, \texttt{R1} cuenta con $2$ interfaces. La primera pertenece a la misma subred que \texttt{SRC} y la segunda a la misma que \texttt{DST}. Para poder controlar la congestión en el encaminador a nuestro antojo modificaremos la velocidad de la interfaz conectada a \texttt{DST} desde el punto de vista de la capa de red. Al estar empleando interfaces \texttt{fastEthernet} podríamos esperar una tasa en las líneas asociadas de $100\ Mbps$. No obstante y debido a las limitaciones intrínsecas a nuestro escenario virtualizado veremos que esas velocidades se ``quedan'' en unos $20\ Mbps$. No especificamos una velocidad ya que la velocidad real variará ligeramente en función de la potencia del equipo sobre el que ejecutemos las simulaciones. En nuestro caso disponemos de $16\ GB$ de memoria \texttt{RAM} y un procesador \textit{Intel(R) Core(TM) i7-5500U CPU @ 2.40GHz} tal y como indica el comando \texttt{lscpu}. Nuestro sistema operativo anfitrión es \texttt{Ubuntu 20.04}.

        \subsection{Configurando carpetas compartidas}
            Antes de proceder a probar el escenario queremos dedicar una pequeña sección a explicar la configuración que hemos llevado a cabo en las máquinas virtuales para poder extraer los archivos que recogen las trazas de tráfico en base a las cuales obtendremos las gráficas que adjuntamos en el documento. Personalmente no nos agrada demasiado trabajar dentro de máquinas virtuales y, al estar ejecutando \texttt{Ubuntu} de forma nativa preferimos generar las gráficas en nuestro propio equipo para poder guardarlas de manera más cómoda.\\

            Para conseguir compartir archivos basta con configurar las carpetas compartidas en cada una de las $3$ máquinas virtuales del entorno. Comenzaremos explicando la jerarquía que debemos crear desde la máquina anfitriona. En el sistema operativo nativo solo debemos crear un directorio en el que se almacenarán todos los archivos generados. En nuestro caso lo hemos nombrado \texttt{VMWare\_shares} y la ruta absoluta hasta él viene dada por \texttt{/home/pablo/VMWare\_shares}. En el \textit{VMWare Player} que ejecutamos en el anfitrión configuraremos una carpeta compartida en el menú \texttt{Virtual Machine > Virtual Machine Settings} al que también podemos acceder con \texttt{CTRL + D}. Una vez dentro hacemos click en \texttt{Options} y creamos una carpeta compartida cuyo \texttt{Host Path} apunte a \texttt{/home/pablo/VMWare\_shares}. Debemos anotar el nombre que le damos al directorio compartido ya que después de crearlo aparecerá la carpeta \texttt{/mnt/hgfs/Foo} donde \texttt{Foo} es el nombre que nosotros le hemos dado a la carpeta compartida en cuestión. A fin de trabajar de manera más cómoda hemos creado un \textit{enalce simbólico} a dicho directorio en el directorio de conexión del usuario \texttt{itraf} en la máquina virtual. En otras palabras, hemos ejecutado \texttt{ln -s /mnt/hgfs/Foo /home/itraf/Host\_shares} con lo que el directorio \texttt{Host\_shares} es un enlace a la carpeta compartida.\\

            En cada una de las máquinas que lanzamos al ejecutar el escenario de \texttt{GNS3} haremos algo parecido. Navegando por \texttt{VM > Settings} (o pulsando de nuevo \texttt{CTRL + D}) configuramos una nueva carpeta compartida en cada máquina. Llamaremos a dicha carpeta \texttt{Foo} de nuevo y la apuntaremos contra el directorio \texttt{/home/itraf/Host\_shares} de la máquina virtual "original". Para trabajar de forma más cómoda ejecutaremos \texttt{ln -s /mnt/hgfs/Foo /home/itraf/Int\_shares} en cada máquina del escenario de \texttt{GNS3} para lograr lo mismo que hacíamos con el directorio \texttt{Host\_shares} en el párrafo anterior.\\

            Tal y como está montado el escenario ahora basta con mover un archivo a la carpeta \texttt{Int\_shares} en cualquiera de las máquinas lanzadas por \texttt{GNS3} para que éste aparezca en la carpeta \texttt{VMWare\_shares} del sistema operativo anterior. Así podemos extraer los archivos necesarios de una manera sencilla para procesarlos más tarde sin tener la monstruosidad de escenario devorando memoria \texttt{RAM} como si no hubiera un mañana.

    \section{Generando tráfico sin saturar el router}
        En una primera instancia intentaremos generar tráfico sin saturar el enlace de salida del router a fin de no provocar que el \textit{buffer} de salida del router se desborde. Para ello debemos recordar que la interfaz de salida del mismo tiene una tasa de alrededor de $20\ Mbps$ tal y como comentábamos anteriormente. Tal y como sugiere el guión de la práctica vamos a configurar \texttt{SRC} para que emita $4$ flujos de datos de tal manera que en el momento de mayor solapamiento solo tengamos una tasa de entrada total de $17\ Mbps$. Para controlar el ``caudal'' de cada flujo podemos alterar el tamaño del paquete en la capa de aplicación (esto es, la \textit{SDU} de la capa de transporte) o el número de paquetes de ese tamaño que la capa de aplicación emite por segundo. Dado que ambos parámetros controlan de manera directa el ``tamaño'' de los flujos fijaremos uno de ellos para variar el otro hasta conseguir el objetivo deseado.\\

        En nuestro caso vamos a fijar el tamaño de la \textit{SDU} en la capa de transporte a $1250\ B$. Aún añadiendo las cabeceras de \texttt{UDP, IP} y \texttt{Ethernet} tenemos un tamaño total en la capa física de $L_{phys} = 1250 + 8 + 20 + 38 = 1316\ B$. Recordemos que la \textit{MTU} típica de \texttt{Ethernet} es de $1500\ B$ con lo que no la estamos superando. De hacerlo deberíamos tener en cuenta el aumento de sobrecarga de cabeceras al que nos enfrentaríamos ya que un mensaje de la capa de aplicación requeriría más de una trama en la capa de enlace.\\

        Si comparamos la sobrecarga introducida por las cabeceras con el tamaño del mensaje veremos que éstas no representan ni siquiera el $5,3\%$ del total del paquete ($\frac{L_{hdr}}{L_{app}} = \frac{8 + 20 + 38}{1250} = 0,0528 < 0,053$). Esto implica que la velocidad en la capa física ($V_{phy}$) será un $5,3\%$ mayor que la de la capa de aplicación ($V_{app}$), que es la que vamos a medir.

        $$\frac{V_{phy}}{V_{app}} = \frac{\frac{L_{phy}}{t}}{\frac{L_{app}}{t}} = \frac{L_{phy}}{L_{app}} = \frac{L_{app} + L_{hdr}}{L_{app}} = 1 + \frac{66\ B}{1250\ B} = 1 + 0,0528 \rightarrow V_{phy} = 1,0528 \cdot V_{app}$$

        Despreciaremos esta diferencia pero es bueno conocer que existe tal y como nos invita a hacer el guión de la práctica.\\

        En cualquier caso con un tamaño de mensaje $L_{app} = 1250\ B$ estamos dentro de los límites de la \textit{MTU} de \texttt{Ethernet} con lo que es apropiado. Una vez decidido esto vamos a ver cómo podemos lograr una velocidad determinada en un flujo variando el número de mensajes por segundo $N$. Aplicando la mismísima definición de tasa:

        \begin{multline*}
            V_e = N\ \frac{pkt}{s}\ \cdot 1250\ \frac{B}{pkt}\ \cdot 8\ \frac{b}{B} \rightarrow V_e = N \cdot 10^4\ bps \rightarrow N = 10^{-4} \cdot V_e\ [bps] \rightarrow\\
            \rightarrow N = 10^{-4} \cdot V_e \cdot 10^6\ \frac{Mbps}{bps} = 100 \cdot V_e\ [Mbps]
        \end{multline*}

        Esto es, que si queremos un flujo de $8\ Mbps$ debemos especificar $N = 100 \cdot 8 = 800\ \frac{pkt}{s}$ en la configuración de \texttt{mgen} en el emisor. Teniendo esto claro vamos a adjuntar cada una de las figuras que se nos requerían en este caso.

        \subsection{Figuras}
            \subsubsection{Tasa de cada flujo}
                

\end{document}
